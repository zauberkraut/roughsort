\documentclass[10pt, xcolor={dvipsnames}, aspectratio=169]{beamer}
\usefonttheme[onlymath]{serif}
\usepackage[utf8]{inputenc}
\usepackage[super, square]{natbib}
\usepackage{mathtools, relsize, graphicx, amssymb, amsthm}
\usepackage{listings}
\usepackage{color}

\lstset{
    frame=single,
    basicstyle=\footnotesize,
    keywordstyle=\color{purple},
    numbers=left,
    numbersep=5pt,
    showstringspaces=false, 
    stringstyle=\color{blue},
    tabsize=4,
    language=C++
}

\mode<presentation>{\usetheme{Berlin} \usecolortheme{beetle}}

\title{Parallel Sorting of Roughly-Sorted Sequences}
\author{Anthony Pfaff, Jason Treadwell}
\institute{CSCI 5172 $|$ CU Denver $|$ Fall '16}
\date{11.??.2016}

\begin{document}
\setbeamertemplate{navigation symbols}{} %remove navigation symbols
\bibliographystyle{plain}
\nocite{*}
% gets rid of the references button
\renewcommand{\bibsection}{\subsubsection*{\bibname}}
\graphicspath{{./}}

\begin{frame}
\titlepage
\end{frame}

\begin{frame}
\transfade
\frametitle{Introduction}
% TODO: rewrite this
Sorting a collection according to some ordering among its items is among the most classic problems of computer science.
A well-established result is the linearithmic  (i.e.\ $O(n \lg n)$) optimal upper bound for sorting sequences of length $n$ by
  comparison.
\end{frame}

\begin{frame}
\frametitle{References}
\smaller[1]{\bibliography{refs}}
\end{frame}

\begin{frame}
\frametitle{Demo}
\end{frame}

\end{document}
