\documentclass[letterpaper, 12pt]{article}
\usepackage[letterpaper]{geometry}
\usepackage{amsmath}
\usepackage{amssymb}
\usepackage[doublespacing]{setspace}
\usepackage{graphicx}
\usepackage{standalone}
\usepackage{url}
\title{Parallel Sorting of Roughly-Sorted Sequences\\CSCI 5172 Fall '16 Project}
\author{Anthony Pfaff, Jason Treadwell}

\begin{document}
\maketitle

% TODO: rewrite away the proposal text
Sorting a collection according to some ordering among its items is among the most classic problems of computer science.
A well-established result is the linearithmic  (i.e.\ $O(n \lg n)$) optimal upper bound for sorting sequences of length $n$ by
  comparison.
Commonly-used sorting algorithms such as Mergesort and Quicksort have $\Theta(n \lg n)$ and $\Omega(n \lg n)$ runtimes,
  respectively.

We can exploit the ordering of \textit{roughly-sorted} sequences to sort them in $O(n \lg k)$ time, where $k$ is the
  \textit{radius} of a sequence $S$ or the smallest $k$ such that $S$ is $k$-\textit{sorted}.\cite{altman89}
A $k$-sorted sequence $\{a_0, a_1, \cdots, a_n\}$ satisfies $a_i \leq a_j \;\forall\; 1 \leq i \leq j \leq n, i \leq j - k$.
Since an unsorted sequence can be at most $n$-sorted, the worst-case runtime of this algorithm has complexity $O(n \lg n)$.

We intend to study and implement sorting of roughly-sorted sequences using both sequential and parallel\cite{altman90} versions
  of the algorithm. In particular, I shall investigate the applicability of the method over very large sequences,
  accelerated using graphics hardware. A GPU is dedicated graphics hardware that also excels at performing frequent, basic
  computations over large amounts of data. The use of GPUs for numerical applications is a recent and rapidly-growing development
  in computer science and software engineering. Suitable hardware is now commonly available in consumer devices and may herald a
  paradigm shift in how we efficiently perform common and indispensable tasks like sorting.

I propose to develop implementations of the method using C and Nvidia's CUDA framework, comparing the results with sequential
  implementations running on both the GPU and CPU as well as with the classic Mergesort and Quicksort algorithms. If time
  permits, I would also like to compare the method to parallel versions of the latter two algorithms.

The aforementioned algorithms assume a specific underlying memory in achieving the provided parallel implementation runtimes: the concurrent-read concurrent-write parallel random access model (CRCW PRAM).  Shiloach and Vishkin elaborate on this memory model, providing that it is a purely theoretical model which, inter alia, meets the following criteria: a shared, global memory of size m\textgreater p is available to p processors, and up to p positions in this memory may be simultaneously read or written in one step\cite{Shiloach1981}.  As with roughsort, a number of algorithms take advantage of this model\textquoteright s assumptions to provide efficiency improvements over their serial counterparts\cite{Shiloach1982}\cite{raj1989}.

Conspicuous, however, with these and other CRCW PRAM algorithms is a lack of presented implementations.  The model is merely that, and thus implementing such an algorithm requires a suitable proxy which shares the characteristics of CRCW PRAM.  In contemporary computing, the most notable such proxy is graphics processing unit (GPU) parallelization.

GPUs invert a core principle of general purpose computing to yield a performance improvement on specific workloads.  Rather than having one or a handful of general purpose, powerful multiprocessors, GPUs opt for a different model: many less powerful, special purpose processors.  Contemporary examples include [Jason's CPU information ] on a CPU, versus [Jason's GPU information] on a GPU.  

Nvidia--the manufacturuer of the aforemeitoned graphics card--makes available an API suite and SDK which allows developers to readily access the GPU: CUDA.  This platform provides developers a simple interface with which they can offload processing of CUDA-augmented C/C++ software payloads to CUDA-supported Nvidia cards.

The parallelism provided by Nvidia and CUDA in this scheme can be realized by developers who develop their payloads in specific ways.  Fundamentally, programs running on an Nvidia card via CUDA operate differently than their  CPU counterparts.  In general, a line of code executing on an Nvidia card is not processed in isolation.  Rather, the Nvidia CUDA scheme provides for single instruction, multiple thread (SIMT) execution, as opposed to the usual single instruction, single data scheme (SISD) provided by general purpose CPUs.  In contrast to a single instruction accessing or updating one memory location on a single core of a general purpose CPU, a single instruction executing on an Nvidia card is executing inside multiple threads simultaneously, each of which operates on a different location in memory.

When a developer wants to take advantage of this parallelism, they structure their code such that the multiple results of the same instruction executing is a useful component of the overall program's payload.  Nvidia CUDA provides simple mechanisms by which to achieve this.

The core component of an Nvidia payload is a thread; an instance of the series of instructions in the overall CUDA payload.  Multiple threads run in a group called a block.  Blocks contain a developer-specified number of threads each.  Between threads and blocks is the concept of a warp.  Warps are collections of 32 threads executing simultaneously on the same processor core on the Nvidia card.  Usually, all threads in a warp are executing the same instruction simultaneously.  Blocks are divvyed up into multiple warps for scheduling an execution on an Nvidia GPU.  Collections of these blocks are referred to as a grid.  Finally, encompassing all of these components is the kernel.  Kernels are the series of instructions which threads, warps, blocks, and grids comprise an instance of and represent the sequence of code to actually be represented on GPU.

Nvidia provides a simple mechanism by which developers can access the parallelism provided by CUDA.  Each of the above divisions of the CUDA payload are provided with an identifier.  A tuple composed of the thread, block, and grid identifier\footnote{Concerns with warps are abstracted away from the developer.  Consequently warp identifiers are not needed and are not available.} uniquely identifies that component of the workload.  Using this information, individual threads all executing the same instruction (such as a memory from an array) can perform that instruction on distinct data (access different memory locations in the array).\cite{nvidiaguide}

Using these techniques provided a platform to develop the GPU parallel versions of sequence radius finding and subsequent roughsorting.  CUDA implementations of the RL, LR, and DM sequences were developed.  This was done by having assigning each thread a unique identifier and then, analagous to Altman and Chlebus\textquoteright work  \cite{altman89}, having teach thread manage the corresponding slot in the global unsorted array.  Each thread performed the parallel min- then max-prefix operations until the respective lists were efficiently determined ordered.  Sorting was determined by having each thread determine if its assigned slot in the LR or RL list was greater than the next element in the list, and having each thread globally assert any true (unsorted) result to global memory, available to all other threads.

Once the RL and LR lists were completed, the DM list was computed by using Altman et. al.\textquoteright s serial radius techniques in conjunction with the corresponding log \cite{altman89}.


Given many GPU processor threads may update multiple memory locations simultaneously, and likewise many memory location may be read in parallel on GPU, the underlying NVIDIA hardware that CUDA provides access to is reminiscent of CRCW PRAM.  However, GPUs differ from true CRCW PRAM in fundamental ways.  First, GPU memory is finite, while PRAM is of arbitrary size.  Furthermore, for GPU memory access times to provide the concurrent read and write characteristics reminiscent of PRAM, these memory operations must be performed on contiguous portions of GPU memory, whereas CRCW PRAM has no such ordering limitation.  Foremost in the differences between CRCW PRAM and GPU architectures however is the warp divergence concept addressed above.  What this implies for memory operations is that workloads exist which may prevent all but one thread in a warp from performing a memory operation at a particular time.  Consequently, there are practical limitations to what a GPU may achieve, and these limitations restrict the workloads which are practical to run on a GPU.  Not only must a GPU workload access constrain its memory usage within practical limitations, the accesses to that memory must be orderly, and the underlying algorithm performing those accesses must be amenable to parallelization on GPU if the advantages of GPU processing over CPU processing are to be realized\cite{dehne2010exploring}.

% GNUPLOT: LaTeX picture
\setlength{\unitlength}{0.240900pt}
\ifx\plotpoint\undefined\newsavebox{\plotpoint}\fi
\sbox{\plotpoint}{\rule[-0.200pt]{0.400pt}{0.400pt}}%
\begin{picture}(2550,1800)(0,0)
\sbox{\plotpoint}{\rule[-0.200pt]{0.400pt}{0.400pt}}%
\put(131.0,131.0){\rule[-0.200pt]{4.818pt}{0.400pt}}
\put(111,131){\makebox(0,0)[r]{$0$}}
\put(2469.0,131.0){\rule[-0.200pt]{4.818pt}{0.400pt}}
\put(131.0,335.0){\rule[-0.200pt]{4.818pt}{0.400pt}}
\put(111,335){\makebox(0,0)[r]{$2$}}
\put(2469.0,335.0){\rule[-0.200pt]{4.818pt}{0.400pt}}
\put(131.0,538.0){\rule[-0.200pt]{4.818pt}{0.400pt}}
\put(111,538){\makebox(0,0)[r]{$4$}}
\put(2469.0,538.0){\rule[-0.200pt]{4.818pt}{0.400pt}}
\put(131.0,742.0){\rule[-0.200pt]{4.818pt}{0.400pt}}
\put(111,742){\makebox(0,0)[r]{$6$}}
\put(2469.0,742.0){\rule[-0.200pt]{4.818pt}{0.400pt}}
\put(131.0,945.0){\rule[-0.200pt]{4.818pt}{0.400pt}}
\put(111,945){\makebox(0,0)[r]{$8$}}
\put(2469.0,945.0){\rule[-0.200pt]{4.818pt}{0.400pt}}
\put(131.0,1149.0){\rule[-0.200pt]{4.818pt}{0.400pt}}
\put(111,1149){\makebox(0,0)[r]{$10$}}
\put(2469.0,1149.0){\rule[-0.200pt]{4.818pt}{0.400pt}}
\put(131.0,1352.0){\rule[-0.200pt]{4.818pt}{0.400pt}}
\put(111,1352){\makebox(0,0)[r]{$12$}}
\put(2469.0,1352.0){\rule[-0.200pt]{4.818pt}{0.400pt}}
\put(131.0,1556.0){\rule[-0.200pt]{4.818pt}{0.400pt}}
\put(111,1556){\makebox(0,0)[r]{$14$}}
\put(2469.0,1556.0){\rule[-0.200pt]{4.818pt}{0.400pt}}
\put(131.0,1759.0){\rule[-0.200pt]{4.818pt}{0.400pt}}
\put(111,1759){\makebox(0,0)[r]{$16$}}
\put(2469.0,1759.0){\rule[-0.200pt]{4.818pt}{0.400pt}}
\put(131.0,131.0){\rule[-0.200pt]{0.400pt}{4.818pt}}
\put(131,90){\makebox(0,0){.025}}
\put(131.0,1739.0){\rule[-0.200pt]{0.400pt}{4.818pt}}
\put(215.0,131.0){\rule[-0.200pt]{0.400pt}{4.818pt}}
\put(215,90){\makebox(0,0){.05}}
\put(215.0,1739.0){\rule[-0.200pt]{0.400pt}{4.818pt}}
\put(299.0,131.0){\rule[-0.200pt]{0.400pt}{4.818pt}}
\put(299,90){\makebox(0,0){.075}}
\put(299.0,1739.0){\rule[-0.200pt]{0.400pt}{4.818pt}}
\put(384.0,131.0){\rule[-0.200pt]{0.400pt}{4.818pt}}
\put(384,90){\makebox(0,0){.1}}
\put(384.0,1739.0){\rule[-0.200pt]{0.400pt}{4.818pt}}
\put(468.0,131.0){\rule[-0.200pt]{0.400pt}{4.818pt}}
\put(468,90){\makebox(0,0){.125}}
\put(468.0,1739.0){\rule[-0.200pt]{0.400pt}{4.818pt}}
\put(552.0,131.0){\rule[-0.200pt]{0.400pt}{4.818pt}}
\put(552,90){\makebox(0,0){.15}}
\put(552.0,1739.0){\rule[-0.200pt]{0.400pt}{4.818pt}}
\put(636.0,131.0){\rule[-0.200pt]{0.400pt}{4.818pt}}
\put(636,90){\makebox(0,0){.175}}
\put(636.0,1739.0){\rule[-0.200pt]{0.400pt}{4.818pt}}
\put(721.0,131.0){\rule[-0.200pt]{0.400pt}{4.818pt}}
\put(721,90){\makebox(0,0){.2}}
\put(721.0,1739.0){\rule[-0.200pt]{0.400pt}{4.818pt}}
\put(805.0,131.0){\rule[-0.200pt]{0.400pt}{4.818pt}}
\put(805,90){\makebox(0,0){.25}}
\put(805.0,1739.0){\rule[-0.200pt]{0.400pt}{4.818pt}}
\put(889.0,131.0){\rule[-0.200pt]{0.400pt}{4.818pt}}
\put(889,90){\makebox(0,0){.3}}
\put(889.0,1739.0){\rule[-0.200pt]{0.400pt}{4.818pt}}
\put(973.0,131.0){\rule[-0.200pt]{0.400pt}{4.818pt}}
\put(973,90){\makebox(0,0){.35}}
\put(973.0,1739.0){\rule[-0.200pt]{0.400pt}{4.818pt}}
\put(1057.0,131.0){\rule[-0.200pt]{0.400pt}{4.818pt}}
\put(1057,90){\makebox(0,0){.4}}
\put(1057.0,1739.0){\rule[-0.200pt]{0.400pt}{4.818pt}}
\put(1142.0,131.0){\rule[-0.200pt]{0.400pt}{4.818pt}}
\put(1142,90){\makebox(0,0){.45}}
\put(1142.0,1739.0){\rule[-0.200pt]{0.400pt}{4.818pt}}
\put(1226.0,131.0){\rule[-0.200pt]{0.400pt}{4.818pt}}
\put(1226,90){\makebox(0,0){.5}}
\put(1226.0,1739.0){\rule[-0.200pt]{0.400pt}{4.818pt}}
\put(1310.0,131.0){\rule[-0.200pt]{0.400pt}{4.818pt}}
\put(1310,90){\makebox(0,0){.6}}
\put(1310.0,1739.0){\rule[-0.200pt]{0.400pt}{4.818pt}}
\put(1394.0,131.0){\rule[-0.200pt]{0.400pt}{4.818pt}}
\put(1394,90){\makebox(0,0){.7}}
\put(1394.0,1739.0){\rule[-0.200pt]{0.400pt}{4.818pt}}
\put(1478.0,131.0){\rule[-0.200pt]{0.400pt}{4.818pt}}
\put(1478,90){\makebox(0,0){.8}}
\put(1478.0,1739.0){\rule[-0.200pt]{0.400pt}{4.818pt}}
\put(1563.0,131.0){\rule[-0.200pt]{0.400pt}{4.818pt}}
\put(1563,90){\makebox(0,0){.9}}
\put(1563.0,1739.0){\rule[-0.200pt]{0.400pt}{4.818pt}}
\put(1647.0,131.0){\rule[-0.200pt]{0.400pt}{4.818pt}}
\put(1647,90){\makebox(0,0){1}}
\put(1647.0,1739.0){\rule[-0.200pt]{0.400pt}{4.818pt}}
\put(1731.0,131.0){\rule[-0.200pt]{0.400pt}{4.818pt}}
\put(1731,90){\makebox(0,0){1.1}}
\put(1731.0,1739.0){\rule[-0.200pt]{0.400pt}{4.818pt}}
\put(1815.0,131.0){\rule[-0.200pt]{0.400pt}{4.818pt}}
\put(1815,90){\makebox(0,0){1.2}}
\put(1815.0,1739.0){\rule[-0.200pt]{0.400pt}{4.818pt}}
\put(1899.0,131.0){\rule[-0.200pt]{0.400pt}{4.818pt}}
\put(1899,90){\makebox(0,0){1.3}}
\put(1899.0,1739.0){\rule[-0.200pt]{0.400pt}{4.818pt}}
\put(1984.0,131.0){\rule[-0.200pt]{0.400pt}{4.818pt}}
\put(1984,90){\makebox(0,0){1.4}}
\put(1984.0,1739.0){\rule[-0.200pt]{0.400pt}{4.818pt}}
\put(2068.0,131.0){\rule[-0.200pt]{0.400pt}{4.818pt}}
\put(2068,90){\makebox(0,0){1.5}}
\put(2068.0,1739.0){\rule[-0.200pt]{0.400pt}{4.818pt}}
\put(2152.0,131.0){\rule[-0.200pt]{0.400pt}{4.818pt}}
\put(2152,90){\makebox(0,0){1.6}}
\put(2152.0,1739.0){\rule[-0.200pt]{0.400pt}{4.818pt}}
\put(2236.0,131.0){\rule[-0.200pt]{0.400pt}{4.818pt}}
\put(2236,90){\makebox(0,0){1.7}}
\put(2236.0,1739.0){\rule[-0.200pt]{0.400pt}{4.818pt}}
\put(2321.0,131.0){\rule[-0.200pt]{0.400pt}{4.818pt}}
\put(2321,90){\makebox(0,0){1.8}}
\put(2321.0,1739.0){\rule[-0.200pt]{0.400pt}{4.818pt}}
\put(2405.0,131.0){\rule[-0.200pt]{0.400pt}{4.818pt}}
\put(2405,90){\makebox(0,0){1.9}}
\put(2405.0,1739.0){\rule[-0.200pt]{0.400pt}{4.818pt}}
\put(2489.0,131.0){\rule[-0.200pt]{0.400pt}{4.818pt}}
\put(2489,90){\makebox(0,0){2}}
\put(2489.0,1739.0){\rule[-0.200pt]{0.400pt}{4.818pt}}
\put(131.0,131.0){\rule[-0.200pt]{0.400pt}{392.185pt}}
\put(131.0,131.0){\rule[-0.200pt]{568.042pt}{0.400pt}}
\put(2489.0,131.0){\rule[-0.200pt]{0.400pt}{392.185pt}}
\put(131.0,1759.0){\rule[-0.200pt]{568.042pt}{0.400pt}}
\put(30,945){\makebox(0,0){(ms)}}
\put(1310,29){\makebox(0,0){$n$}}
\put(431,1708){\makebox(0,0)[r]{   Seq. Radius}}
\put(451.0,1708.0){\rule[-0.200pt]{24.090pt}{0.400pt}}
\put(131,131){\usebox{\plotpoint}}
\multiput(552.58,131.00)(0.499,0.607){165}{\rule{0.120pt}{0.586pt}}
\multiput(551.17,131.00)(84.000,100.784){2}{\rule{0.400pt}{0.293pt}}
\put(131.0,131.0){\rule[-0.200pt]{101.419pt}{0.400pt}}
\multiput(889.58,233.00)(0.499,0.607){165}{\rule{0.120pt}{0.586pt}}
\multiput(888.17,233.00)(84.000,100.784){2}{\rule{0.400pt}{0.293pt}}
\put(636.0,233.0){\rule[-0.200pt]{60.948pt}{0.400pt}}
\multiput(1226.58,335.00)(0.499,0.601){165}{\rule{0.120pt}{0.581pt}}
\multiput(1225.17,335.00)(84.000,99.794){2}{\rule{0.400pt}{0.290pt}}
\put(973.0,335.0){\rule[-0.200pt]{60.948pt}{0.400pt}}
\multiput(1394.58,436.00)(0.499,0.607){165}{\rule{0.120pt}{0.586pt}}
\multiput(1393.17,436.00)(84.000,100.784){2}{\rule{0.400pt}{0.293pt}}
\put(1310.0,436.0){\rule[-0.200pt]{20.236pt}{0.400pt}}
\multiput(1563.58,538.00)(0.499,0.607){165}{\rule{0.120pt}{0.586pt}}
\multiput(1562.17,538.00)(84.000,100.784){2}{\rule{0.400pt}{0.293pt}}
\multiput(1647.58,640.00)(0.499,0.607){165}{\rule{0.120pt}{0.586pt}}
\multiput(1646.17,640.00)(84.000,100.784){2}{\rule{0.400pt}{0.293pt}}
\put(1478.0,538.0){\rule[-0.200pt]{20.476pt}{0.400pt}}
\multiput(1815.58,742.00)(0.499,0.601){165}{\rule{0.120pt}{0.581pt}}
\multiput(1814.17,742.00)(84.000,99.794){2}{\rule{0.400pt}{0.290pt}}
\put(1731.0,742.0){\rule[-0.200pt]{20.236pt}{0.400pt}}
\multiput(1984.58,843.00)(0.499,0.607){165}{\rule{0.120pt}{0.586pt}}
\multiput(1983.17,843.00)(84.000,100.784){2}{\rule{0.400pt}{0.293pt}}
\put(1899.0,843.0){\rule[-0.200pt]{20.476pt}{0.400pt}}
\multiput(2152.58,945.00)(0.499,0.607){165}{\rule{0.120pt}{0.586pt}}
\multiput(2151.17,945.00)(84.000,100.784){2}{\rule{0.400pt}{0.293pt}}
\put(2068.0,945.0){\rule[-0.200pt]{20.236pt}{0.400pt}}
\multiput(2321.58,1047.00)(0.499,0.607){165}{\rule{0.120pt}{0.586pt}}
\multiput(2320.17,1047.00)(84.000,100.784){2}{\rule{0.400pt}{0.293pt}}
\multiput(2405.58,1149.00)(0.499,0.601){165}{\rule{0.120pt}{0.581pt}}
\multiput(2404.17,1149.00)(84.000,99.794){2}{\rule{0.400pt}{0.290pt}}
\put(131,131){\makebox(0,0){$\bullet$}}
\put(215,131){\makebox(0,0){$\bullet$}}
\put(299,131){\makebox(0,0){$\bullet$}}
\put(384,131){\makebox(0,0){$\bullet$}}
\put(468,131){\makebox(0,0){$\bullet$}}
\put(552,131){\makebox(0,0){$\bullet$}}
\put(636,233){\makebox(0,0){$\bullet$}}
\put(721,233){\makebox(0,0){$\bullet$}}
\put(805,233){\makebox(0,0){$\bullet$}}
\put(889,233){\makebox(0,0){$\bullet$}}
\put(973,335){\makebox(0,0){$\bullet$}}
\put(1057,335){\makebox(0,0){$\bullet$}}
\put(1142,335){\makebox(0,0){$\bullet$}}
\put(1226,335){\makebox(0,0){$\bullet$}}
\put(1310,436){\makebox(0,0){$\bullet$}}
\put(1394,436){\makebox(0,0){$\bullet$}}
\put(1478,538){\makebox(0,0){$\bullet$}}
\put(1563,538){\makebox(0,0){$\bullet$}}
\put(1647,640){\makebox(0,0){$\bullet$}}
\put(1731,742){\makebox(0,0){$\bullet$}}
\put(1815,742){\makebox(0,0){$\bullet$}}
\put(1899,843){\makebox(0,0){$\bullet$}}
\put(1984,843){\makebox(0,0){$\bullet$}}
\put(2068,945){\makebox(0,0){$\bullet$}}
\put(2152,945){\makebox(0,0){$\bullet$}}
\put(2236,1047){\makebox(0,0){$\bullet$}}
\put(2321,1047){\makebox(0,0){$\bullet$}}
\put(2405,1149){\makebox(0,0){$\bullet$}}
\put(2489,1250){\makebox(0,0){$\bullet$}}
\put(501,1708){\makebox(0,0){$\bullet$}}
\put(2236.0,1047.0){\rule[-0.200pt]{20.476pt}{0.400pt}}
\put(431,1646){\makebox(0,0)[r]{   Par. Radius}}
\put(451.0,1646.0){\rule[-0.200pt]{24.090pt}{0.400pt}}
\put(131,131){\usebox{\plotpoint}}
\multiput(131.58,131.00)(0.499,0.607){165}{\rule{0.120pt}{0.586pt}}
\multiput(130.17,131.00)(84.000,100.784){2}{\rule{0.400pt}{0.293pt}}
\multiput(636.58,233.00)(0.499,0.600){167}{\rule{0.120pt}{0.580pt}}
\multiput(635.17,233.00)(85.000,100.796){2}{\rule{0.400pt}{0.290pt}}
\put(215.0,233.0){\rule[-0.200pt]{101.419pt}{0.400pt}}
\multiput(889.58,335.00)(0.499,0.601){165}{\rule{0.120pt}{0.581pt}}
\multiput(888.17,335.00)(84.000,99.794){2}{\rule{0.400pt}{0.290pt}}
\put(721.0,335.0){\rule[-0.200pt]{40.471pt}{0.400pt}}
\multiput(1142.58,436.00)(0.499,0.607){165}{\rule{0.120pt}{0.586pt}}
\multiput(1141.17,436.00)(84.000,100.784){2}{\rule{0.400pt}{0.293pt}}
\multiput(1226.58,538.00)(0.499,0.607){165}{\rule{0.120pt}{0.586pt}}
\multiput(1225.17,538.00)(84.000,100.784){2}{\rule{0.400pt}{0.293pt}}
\put(973.0,436.0){\rule[-0.200pt]{40.712pt}{0.400pt}}
\multiput(1394.58,640.00)(0.499,0.607){165}{\rule{0.120pt}{0.586pt}}
\multiput(1393.17,640.00)(84.000,100.784){2}{\rule{0.400pt}{0.293pt}}
\multiput(1478.58,742.00)(0.499,0.594){167}{\rule{0.120pt}{0.575pt}}
\multiput(1477.17,742.00)(85.000,99.806){2}{\rule{0.400pt}{0.288pt}}
\multiput(1563.58,843.00)(0.499,0.607){165}{\rule{0.120pt}{0.586pt}}
\multiput(1562.17,843.00)(84.000,100.784){2}{\rule{0.400pt}{0.293pt}}
\multiput(1647.58,945.00)(0.499,0.607){165}{\rule{0.120pt}{0.586pt}}
\multiput(1646.17,945.00)(84.000,100.784){2}{\rule{0.400pt}{0.293pt}}
\put(1310.0,640.0){\rule[-0.200pt]{20.236pt}{0.400pt}}
\multiput(1815.58,1047.00)(0.499,0.607){165}{\rule{0.120pt}{0.586pt}}
\multiput(1814.17,1047.00)(84.000,100.784){2}{\rule{0.400pt}{0.293pt}}
\multiput(1899.58,1149.00)(0.499,0.594){167}{\rule{0.120pt}{0.575pt}}
\multiput(1898.17,1149.00)(85.000,99.806){2}{\rule{0.400pt}{0.288pt}}
\multiput(1984.58,1250.00)(0.499,0.607){165}{\rule{0.120pt}{0.586pt}}
\multiput(1983.17,1250.00)(84.000,100.784){2}{\rule{0.400pt}{0.293pt}}
\put(1731.0,1047.0){\rule[-0.200pt]{20.236pt}{0.400pt}}
\multiput(2152.58,1352.00)(0.499,0.607){165}{\rule{0.120pt}{0.586pt}}
\multiput(2151.17,1352.00)(84.000,100.784){2}{\rule{0.400pt}{0.293pt}}
\multiput(2236.58,1454.00)(0.499,0.600){167}{\rule{0.120pt}{0.580pt}}
\multiput(2235.17,1454.00)(85.000,100.796){2}{\rule{0.400pt}{0.290pt}}
\multiput(2321.58,1556.00)(0.499,0.601){165}{\rule{0.120pt}{0.581pt}}
\multiput(2320.17,1556.00)(84.000,99.794){2}{\rule{0.400pt}{0.290pt}}
\multiput(2405.58,1657.00)(0.499,0.607){165}{\rule{0.120pt}{0.586pt}}
\multiput(2404.17,1657.00)(84.000,100.784){2}{\rule{0.400pt}{0.293pt}}
\put(131,131){\makebox(0,0){$\circ$}}
\put(215,233){\makebox(0,0){$\circ$}}
\put(299,233){\makebox(0,0){$\circ$}}
\put(384,233){\makebox(0,0){$\circ$}}
\put(468,233){\makebox(0,0){$\circ$}}
\put(552,233){\makebox(0,0){$\circ$}}
\put(636,233){\makebox(0,0){$\circ$}}
\put(721,335){\makebox(0,0){$\circ$}}
\put(805,335){\makebox(0,0){$\circ$}}
\put(889,335){\makebox(0,0){$\circ$}}
\put(973,436){\makebox(0,0){$\circ$}}
\put(1057,436){\makebox(0,0){$\circ$}}
\put(1142,436){\makebox(0,0){$\circ$}}
\put(1226,538){\makebox(0,0){$\circ$}}
\put(1310,640){\makebox(0,0){$\circ$}}
\put(1394,640){\makebox(0,0){$\circ$}}
\put(1478,742){\makebox(0,0){$\circ$}}
\put(1563,843){\makebox(0,0){$\circ$}}
\put(1647,945){\makebox(0,0){$\circ$}}
\put(1731,1047){\makebox(0,0){$\circ$}}
\put(1815,1047){\makebox(0,0){$\circ$}}
\put(1899,1149){\makebox(0,0){$\circ$}}
\put(1984,1250){\makebox(0,0){$\circ$}}
\put(2068,1352){\makebox(0,0){$\circ$}}
\put(2152,1352){\makebox(0,0){$\circ$}}
\put(2236,1454){\makebox(0,0){$\circ$}}
\put(2321,1556){\makebox(0,0){$\circ$}}
\put(2405,1657){\makebox(0,0){$\circ$}}
\put(2489,1759){\makebox(0,0){$\circ$}}
\put(501,1646){\makebox(0,0){$\circ$}}
\put(2068.0,1352.0){\rule[-0.200pt]{20.236pt}{0.400pt}}
\put(131.0,131.0){\rule[-0.200pt]{0.400pt}{392.185pt}}
\put(131.0,131.0){\rule[-0.200pt]{568.042pt}{0.400pt}}
\put(2489.0,131.0){\rule[-0.200pt]{0.400pt}{392.185pt}}
\put(131.0,1759.0){\rule[-0.200pt]{568.042pt}{0.400pt}}
\end{picture}


\bibliographystyle{plain}
\nocite{*}
\bibliography{refs} % bib database

\end{document}
